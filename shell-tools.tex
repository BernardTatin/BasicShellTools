\documentclass{scrartcl}% ===> this file was generated automatically by noweave --- better not edit it

%% prelude.tex
%% https://tex.stackexchange.com/questions/36609/formatting-section-titles
%% KOMA needs it
\let\sf\sffamily
\let\bf\bfseries
\let\tt\ttfamily
\let\it\itshape
\let\rm\rmfamily

\usepackage{boolexpr,pdftexcmds,trace}
\usepackage{multicol}

%% KOMA for headers :
\usepackage{scrlayer-scrpage}
\pagestyle{scrheadings}
\automark[section]{section}
\cfoot[\pagemark]{\pagemark}
%% for date/time
\usepackage{scrdate}
\usepackage{scrtime}

%% package for noweb
\usepackage{common/newnoweb}
\noweboptions{smallcode,longchunks,longxref}

%% fonts and encoding
\usepackage[T1]{fontenc}
%\usepackage[default]{gillius}
\usepackage[condensed,sfdefault]{universalis}
\usepackage{CormorantGaramond}
% \usepackage{quattrocento}
\usepackage[zerostyle=d]{newtxtt}
\renewcommand{\familydefault}{\rmdefault}

\usepackage[utf8]{inputenc}
% Babel for French is known to break KOMA scripts
% \usepackage[french]{babel}

%% packages needed for section coloring
\usepackage[table]{xcolor}
\usepackage{tikz}

%% hyperlinks initialisation
\definecolor{mygray}{rgb}{0.4,0.4,0.4}
\usepackage[bookmarks,backref=page,linkcolor=mygray]{hyperref} %,colorlinks
\hypersetup{%
  pdfauthor = {Bernard Tatin},
  pdftitle = {},
  pdfsubject = {},
  pdfkeywords = {},
  colorlinks=true,
  linkcolor= mygray,
  citecolor= black,
  pageanchor=true,
  urlcolor = mygray,
  plainpages = false,
  linktocpage
}

%% a better abstract (I hope...)
\renewenvironment{abstract}[1]{%
  \hfill
  \begin{minipage}
    {0.95\textwidth}
    \rule{\textwidth}{1pt}
    \footnotesize
    #1\footnote{Document crée le \today à \thistime}
    \normalsize
    {%
      \par\noindent
      \rule{\textwidth}{1pt}
    }
  \end{minipage}
  \clearpage
}
%% format des paragraphes
%% \setlength{\parindent}{0cm}
%% \setlength{\parskip}{4mm}
%% \linespread{1.1}
%% \let\nwdocspar=\smallbreak

%% better itemize
\newenvironment{packed_itemize}{
\begin{itemize}
  \setlength{\itemsep}{0pt}
  \setlength{\parskip}{0pt}
  \setlength{\parsep}{0pt}
}{\end{itemize}}

\renewcommand{\thesection}{\thepart.\arabic{section}}
% from : https://tex.stackexchange.com/questions/245979/numberwithinequationsubsection-fails-for-subsections-0
% reset section counter at each part
\makeatletter
\@addtoreset{section}{part}
\makeatother
%% from https://tex.stackexchange.com/questions/192838/part-heading-style?rq=1
\newcommand{\addtocontenttable}[4]{%
  \refstepcounter{#1}%
  \addcontentsline{toc}{#1}{#2\hspace{#3}#4}%
  % pour les entêtes de page
  % \markboth{}{}%
  % \reset@font
  % \parindent \z@
  % \vspace*{10\p@}%
}

%% for vbox, see https://tex.stackexchange.com/questions/83930/what-are-the-different-kinds-of-boxes-in-latex
\makeatletter
\def\@part[#1]#2{%
\cleardoublepage
  \addtocontenttable{part}{\thepart}{1em}{#1}
  \mbox{%
    \begin{minipage}{7mm}
      \begin{tabular}{@{}p{7mm}@{}}
        \makebox[7mm]{\cellcolor{orange}\Huge\color{white}\bfseries\strut\thepart\rule[-4cm]{0pt}{4cm}}%
      \end{tabular}%
    \end{minipage}
    \kern-2pt
    \begin{minipage}{170mm}
       \tabular[t]{@{}p{1cm}p{\dimexpr\hsize-2.1cm}@{}}\hline
          & \Huge\bfseries\itshape\rule{0pt}{1.5\ht\strutbox}#1\endtabular
    \end{minipage}
    } %% mbox
  % \cleardoublepage
 \vskip 100\p@
} %% \def\@part
\makeatother

\makeatletter
\def\@section[#1]#2{%
% \newcommand{\section}[2]{%
  \cleardoublepage
  \addtocontenttable{section}{\thesection}{3em}{#1}
  \mbox{%
    \begin{minipage}{7mm}
      \begin{tabular}{@{}p{7mm}@{}}
        \makebox[7mm]{\cellcolor{orange}\Large\color{white}\bfseries\strut\thesection{}\rule[-4cm]{0pt}{4cm}}%
      \end{tabular}%
    \end{minipage}
    \kern-2pt
    \begin{minipage}{170mm}
       \tabular[t]{@{}p{1cm}p{\dimexpr\hsize-2.1cm}@{}}\hline
          & \Large\bfseries\itshape\rule{0pt}{1.5\ht\strutbox}#1\endtabular
    \end{minipage}
    } %% mbox
%  \vskip 100\p@
} %% \def\@section
\makeatother

\makeatletter
\long\def\isequal#1#2{\pdf@strcmp{#1}{#2}}
\makeatother

%% colorized sections
\setkomafont{section}{\mysection}
\newcommand{\mysection}[1]{%
    \LARGE\sffamily\bfseries%
    \color{white}%
    \setlength{\fboxsep}{0cm}%already boxed
  \mbox{%
    \begin{minipage}{7mm}
      \begin{tabular}{@{}p{7mm}@{}}
        \makebox[7mm]{\cellcolor{orange}\Large\color{white}\bfseries\strut
        \switch
          \case{\isequal{\thesection}{.0}}
            {}
          \otherwise
            \thesection{}
        \endswitch
        \rule[-1.5cm]{0pt}{4cm}}%
      \end{tabular}%
    \end{minipage}
    \kern-2pt
    \begin{minipage}{170mm}
      \color{mygray}
       \tabular[t]{@{}p{1cm}p{\dimexpr\hsize-2.1cm}@{}}\hline
          & \Large\bfseries\itshape\rule{0pt}{1.5\ht\strutbox}#1\endtabular
    \end{minipage}
    } %% mbox
    % \colorbox{orange!80}{%
    %     \begin{minipage}{\linewidth}%
    %         \vspace*{4pt}%Space before
    %         \hspace*{3pt}
    %         #1
    %         \vspace*{4pt}%Space after
    %     \end{minipage}%
    %     % \ohead{#1}
    % }
    }

%% colorized subsections
\setkomafont{subsection}{\submysection}
\newcommand{\submysection}[1]{%
    \large\sffamily%
    \color{white}%
    \setlength{\fboxsep}{0cm}%already boxed
    \colorbox{orange!80}{%
        \begin{minipage}{\linewidth}%
            \vspace*{4pt}%Space before
            \hspace*{3pt}
            #1
            \vspace*{4pt}%Space after
        \end{minipage}%
    }}

%% colorized subsubsections
\setkomafont{subsubsection}{\subsubmysection}
\newcommand{\subsubmysection}[1]{%
    \itshape\sffamily%
    \color{white}%
    \setlength{\fboxsep}{0cm}%already boxed
    \colorbox{orange!80}{%
        \begin{minipage}{\linewidth}%
            \vspace*{4pt}%Space before
            \hspace*{3pt}
            #1
            \vspace*{4pt}%Space after
        \end{minipage}%
    }}


\author{Bernard Tatin}
\date{2017}
\title{Basic Shell Tools, des outils en shell}
\begin{document}

%% \pagestyle{noweb}
\maketitle
\abstract{Voici des outils plus ou moins simples, écrits en shell \emph{bash} en général car il est certainement le plus répandu actuellement. Certains ne trouveront d'intérêt que pour moi tandis que d'autres pourront satisfaire un plus grand nombre de lecteurs.\\
\\
Le choix de `noweb` provient du simple fait que documentation et sources sont conçus en même temps.}

\tableofcontents
\part{introduction}

L'arborescence des scripts est simple, nous avons un répertoire `bin` qui contient les scripts à exécuter, un répertoire `include` qui contient les scripts de configuration globale à sourcer. Le reste sert \emph{simplement} à gérer la documentation et les tests.

\section{la licence}
C'est la license du \emph{MIT} :
\nwfilename{_noweb.nw}\nwbegincode{1}\sublabel{NW48t7P8-1BEsuV-1}\nwmargintag{{\nwtagstyle{}\subpageref{NW48t7P8-1BEsuV-1}}}\moddef{license.sh~{\nwtagstyle{}\subpageref{NW48t7P8-1BEsuV-1}}}\endmoddef\nwstartdeflinemarkup\nwusesondefline{\\{NW48t7P8-231JFl-1}}\nwenddeflinemarkup
## The MIT License (MIT)
##
## Copyright (c) 2017 Bernard Tatin
##
## Permission is hereby granted, free of charge, to any person obtaining a copy
## of this software and associated documentation files (the "Software"), to deal
## in the Software without restriction, including without limitation the rights
## to use, copy, modify, merge, publish, distribute, sublicense, and/or sell
## copies of the Software, and to permit persons to whom the Software is
## furnished to do so, subject to the following conditions:
##
## The above copyright notice and this permission notice shall be included in all
## copies or substantial portions of the Software.
##
## THE SOFTWARE IS PROVIDED "AS IS", WITHOUT WARRANTY OF ANY KIND, EXPRESS OR
## IMPLIED, INCLUDING BUT NOT LIMITED TO THE WARRANTIES OF MERCHANTABILITY,
## FITNESS FOR A PARTICULAR PURPOSE AND NON INFRINGEMENT. IN NO EVENT SHALL THE
## AUTHORS OR COPYRIGHT HOLDERS BE LIABLE FOR ANY CLAIM, DAMAGES OR OTHER
## LIABILITY, WHETHER IN AN ACTION OF CONTRACT, TORT OR OTHERWISE, ARISING FROM,
## OUT OF OR IN CONNECTION WITH THE SOFTWARE OR THE USE OR OTHER DEALINGS IN THE
## SOFTWARE.
##
\nwused{\\{NW48t7P8-231JFl-1}}\nwendcode{}\nwbegindocs{2}\nwdocspar

\part{le répertoire `include`}

Les fichiers de ce répertoire ne sont utiles que lorsqu'ils sont \emph{sourcés}, soit depuis un shell interactif, soit depuis un script. Pour les shells interactifs, il est conseillé de bien vérifier que tout ce qui est nécessaire est bien exporté.

\section{le fichier `common-bash.bash`}
Pour le moment, c'est le seul fichier. Il permet d'exporter une variable, le nom du script :
\nwenddocs{}\nwbegincode{3}\sublabel{NW48t7P8-2P4XCZ-1}\nwmargintag{{\nwtagstyle{}\subpageref{NW48t7P8-2P4XCZ-1}}}\moddef{common-bash variable~{\nwtagstyle{}\subpageref{NW48t7P8-2P4XCZ-1}}}\endmoddef\nwstartdeflinemarkup\nwusesondefline{\\{NW48t7P8-231JFl-1}}\nwenddeflinemarkup
export script_name=$(basename $0)
\nwused{\\{NW48t7P8-231JFl-1}}\nwendcode{}\nwbegindocs{4}\nwdocspar

Ensuite, nous avons une série de fonctions très utiles:
\nwenddocs{}\nwbegincode{5}\sublabel{NW48t7P8-2KJFr5-1}\nwmargintag{{\nwtagstyle{}\subpageref{NW48t7P8-2KJFr5-1}}}\moddef{common-bash functions~{\nwtagstyle{}\subpageref{NW48t7P8-2KJFr5-1}}}\endmoddef\nwstartdeflinemarkup\nwusesondefline{\\{NW48t7P8-231JFl-1}}\nwenddeflinemarkup
function onerror() \{
    local exit_code=$1
    shift
    local error_msg="$*"

    echo "ERROR: $error_msg" 1>&2
    exit $exit_code
\}

function safe_source() \{
  local file=
  while [ $# -gt 0 ]
  do
  file=$1
    if [ -f $\{file\} ]
    then
      source $\{file\}
    else
      onerror 1 "Cannot source $\{file\}"
    fi
    shift
  done
\}

function get_tmp_file() \{
  local root_name='another-tmp-file'
  [ $# -gt 0 ] && \\
    root_name=$1
  $(mktemp /tmp/$\{root_name\}.XXXXXX)
\}

function dohelp() \{
  local exit_value=0
  local error_message=

  case "$#" in
    '1')
      exit_value=$1
      ;;
    '2')
      exit_value=$1
      shift
      error_message="$*"
      ;;
  esac
  [ -n "$error_message" ] && \\
    echo "ERROR : @" 1>&2

  /usr/bin/printf "$\{help_text\}"

  exit $\{exit_value\}
\}
\nwused{\\{NW48t7P8-231JFl-1}}\nwendcode{}\nwbegindocs{6}\nwdocspar

Ce qui nous donne au final:
\nwenddocs{}\nwbegincode{7}\sublabel{NW48t7P8-231JFl-1}\nwmargintag{{\nwtagstyle{}\subpageref{NW48t7P8-231JFl-1}}}\moddef{common-bash.bash~{\nwtagstyle{}\subpageref{NW48t7P8-231JFl-1}}}\endmoddef\nwstartdeflinemarkup\nwenddeflinemarkup
#!/usr/bin/env bash

\LA{}license.sh~{\nwtagstyle{}\subpageref{NW48t7P8-1BEsuV-1}}\RA{}

\LA{}common-bash variable~{\nwtagstyle{}\subpageref{NW48t7P8-2P4XCZ-1}}\RA{}

\LA{}common-bash functions~{\nwtagstyle{}\subpageref{NW48t7P8-2KJFr5-1}}\RA{}
\nwnotused{common-bash.bash}\nwendcode{}\nwbegindocs{8}\nwdocspar
\part{annexes}
\section{la ligne de commande}
Pour obtenir le fichier \LaTeX{} et le code source, voici ce qu'il faut faire depuis un terminal :

\nwenddocs{}\nwbegincode{9}\sublabel{NW48t7P8-2CUXU7-1}\nwmargintag{{\nwtagstyle{}\subpageref{NW48t7P8-2CUXU7-1}}}\moddef{command-line~{\nwtagstyle{}\subpageref{NW48t7P8-2CUXU7-1}}}\endmoddef\nwstartdeflinemarkup\nwenddeflinemarkup
# fichier LaTeX
noweave -delay -autodefs c -index rbuffer.nw > rbuffer.tex
# fichier PDF
pdflatex rbuffer.tex && \\
  pdflatex rbuffer.tex && \\
  pdflatex rbuffer.tex
# le code source
notangle rbuffer.nw > rbuffer.h
\nwnotused{command-line}\nwendcode{}

\nwixlogsorted{c}{{command-line}{NW48t7P8-2CUXU7-1}{\nwixd{NW48t7P8-2CUXU7-1}}}%
\nwixlogsorted{c}{{common-bash functions}{NW48t7P8-2KJFr5-1}{\nwixd{NW48t7P8-2KJFr5-1}\nwixu{NW48t7P8-231JFl-1}}}%
\nwixlogsorted{c}{{common-bash variable}{NW48t7P8-2P4XCZ-1}{\nwixd{NW48t7P8-2P4XCZ-1}\nwixu{NW48t7P8-231JFl-1}}}%
\nwixlogsorted{c}{{common-bash.bash}{NW48t7P8-231JFl-1}{\nwixd{NW48t7P8-231JFl-1}}}%
\nwixlogsorted{c}{{license.sh}{NW48t7P8-1BEsuV-1}{\nwixd{NW48t7P8-1BEsuV-1}\nwixu{NW48t7P8-231JFl-1}}}%
\nwbegindocs{10}\nwdocspar
L'option {\Tt{}-autodefs\ c\nwendquote} permet à {\Tt{}noweave\nwendquote} de déterminer lui-même les éléments du langage C. Sans cette option, dans le cadre de ce fichier, les définitions de {\Tt{}intro-bits\nwendquote} ne seraient pas visibles.

\section{tables et index}
\subsection{table des extraits de code}

\begin{multicols}{2}
\nowebchunks
\end{multicols}

\subsection{index}

\begin{multicols}{2}
\nowebindex
\end{multicols}

\end{document}
\nwenddocs{}
